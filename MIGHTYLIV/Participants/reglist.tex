\documentclass{article}

\usepackage{lipsum}

\newenvironment{absquote}
               {{\center\bfseries Abstract\endcenter}%
                 \list{}{\leftmargin2cm\rightmargin\leftmargin}%
                \item\relax\footnotesize}
               {\endlist}
\def\author#1{\center#1\endcenter}
\def\articletitle#1{\center{\bfseries\LARGE{#1}}\endcenter}
\def\serial#1{{\bfseries\hfill#1\hspace{1em}}}

\begin{document}

\articletitle{Some wonderful article}

\author{Yiannis Lazarides}

\begin{absquote}
\lipsum[1]
\serial{A-213}
\end{absquote}


first_name: Terry
last_name: McKee
email: terry.mckee@wright.edu
institution: Wright State University
Talk: Yes
Title: Weighty Characterizations of Two Graph Classes
Abstract: Defining the weight of a complete subgraph to be the number of maxcliques that contain it, I give a stylish characterization of strongly chordal graphs in terms of their clique graphs, and a related-but-quirky characterization of trivially perfect graphs.

Sun Feb  3 11:00:33 EST 2013

first_name: David
last_name: Smith
email: smithdaa@gvsu.edu
institution: Grand Valley State University
Talk: no
Title:
Abstract:
Wed Feb  6 10:31:42 EST 2013

first_name: Mingquan
last_name: Zhan
email: Mingquan.Zhan@millersville.edu
institution: Millersville University
Talk: Yes
Title: The Discharging Method and 3-Connected Essentially 10-Connected Line Graphs
Abstract: We use the discharging method to prove that every 3-connected, essentially 10-connected line graph is hamiltonian connected.

Sun Feb 10 23:29:29 EST 2013


first_name: Michael
last_name: Santana
email: santana@illinois.edu
institution: Univeristy of Illinois at Urbana-Champaign
Talk: Yes
Title: Pairs of forbidden subgraphs for pancyclicity
Abstract: The Matthew-Sumner Conjecture claims that every 4-connected, claw-free graph contains a hamiltonian cycle.  This conjecture has provided the impetus for a great deal of research into structural properties of claw-free graphs, and in particular, pancyclicity.  While it is known that forbidding the claw is an insufficient condition to imply pancyclicity of 4-connected graphs, we look to forbid more substructures to obtain this end.  In this talk we will present a characterization of the pairs of forbidden subgraphs that imply pancyclicity for 4-connected graphs.  This extends a result of Gould, {\L}uczak, and Pfender for 3-connected graphs.  (Joint work with James Carraher, Michael Ferrara, and Timothy Morris)
Mon Feb 18 18:37:05 EST 2013

first_name: Xiangqian  "Joe"
last_name: Zhou
email: xzhou@math.wright.edu
institution: Wright State University
Talk: Yes
Title: Unvoidable minors of large 4-connected bicircular matroids
Abstract: The bicircular matroid of a graph G is the matroid with ground set E(G) and a subset is independent iff each connected component of the subgraph spanned by that set contains at most one cycle.   In this talk, we present a recent result on unavoidable minors of large 4-connected bicircular matroids. This is joint work with Daniel Slilaty, Deborah Chun, and Tyler Moss.
Sun Feb 24 10:14:28 EST 2013

first_name: Hong-Jian
last_name: Lai
email: hjlai@#math.wvu.edu
institution: West Virginia University
Talk: no
Title:
Abstract:
Sun Feb 24 11:57:14 EST 2013

first_name: keke
last_name: wang
email: wangkk@math.wvu.edu
institution: West Virginia University
Talk: Yes
Title: Cycle chains and Hamiltonian 3-connected claw-free graphs
Abstract: We develop a cycle chain method to prove that every 3-edge-connected graph G is supereulerian if every 3-edge cut of G intersects with short cycles in G. This is applied to the study of Hamiltonian claw-free graphs, and provide a unified treatment with short proofs for several former results. New sufficient conditions for Hamiltonian claw-free graphs are also obtained.
Mon Feb 25 12:22:46 EST 2013

first_name: Meng
last_name: Zhang
email: zmwvu@math.wvu.edu
institution: 7205 Univerisity Commons Drive, Morgantown, WV, 26505; 304-777-9323
Talk: Yes
Title: Supereulerian Graphs and The Petersen Graph
Abstract: A graph G is supereulerian if G has a spanning eulerian subgraph. Boesch et al proposed the problem of characterizing supereulerian graphs. In this talk I will present any 3-edge-connected graph with at most 11 edgecuts of size 3 is supereulerian if and only if it cannot be contractible to the Petersen graph.
Mon Feb 25 15:12:30 EST 2013

first_name: Ye
last_name: Chen
email: chenyewv@gmail.com
institution: Student
Talk: Yes
Title: r-hued Coloring of K4-minor Free Graphs
Abstract: A list assignment $L$ of $G$ is a mapping that assigns every vertex $v\in V(G)$ a set $L(v)$ of positive integers. For a given list assignment $L$ of $G$, an ($L,r$)-coloring of $G$ is a proper coloring $c$ such that for any vertex $v$ with degree $d(v)$, $c(v)\in L(v)$ and $v$ is adjacent to at least min$\{d(v),r\}$ different colors. The {\it $r$-hued chromatic number} of $G$, $\chi_{r}(G)$, is the least integer $k$ such that for any $v \in V(G)$ with $L(v) = \{1,2,\cdots,k\}$, $G$ has an $(L,r)$-coloring. The {\it $r$-hued list chromatic number} of $G$, $\chi_{L,r}(G)$, is the least integer $k$ such that for any $v \in V(G)$ and every list assignment $L$ with $|L(v)| = k$, $G$ has an $(L,r)$-coloring. Let $K(r)=r+3$ if $2 \le r \le 3$, and  $K(r)=\lfloor 3r/2\rfloor+1$ if $r\ge 4$. We proved that if $G$ is a $K_4$-minor free graph, then

(i) $\chi_r(G) \le K(r)$, and the bound can be attained;

(ii) $\chi_{L,r}(G) \le K(r)+1$.

Mon Feb 25 16:02:51 EST 2013

first_name: Meng
last_name: Zhang
email: mzhang@math.wvu.edu
institution: Graduate student of West Virginia University
Talk: Yes
Title: Supereulerian Graphs and The Petersen Graph
Abstract: A graph G is supereulerian if G has a spanning eulerian subgraph. Boesch et al proposed the problem of characterizing supereulerian graphs. In this talk I will present that any 3-edge-connected graph with at most 11 edgecuts of size 3 is supereulerian if and only if it cannot be contractible to the Petersen graph.
Wed Feb 27 15:04:25 EST 2013


first_name: Gregory
last_name: Puleo
email: gpuleo@gmail.com
institution: University of Illinois at Urbana-Champaign
Talk: Yes
Title: Tuza's Conjecture for Graphs of Max Average Degree Less than 7
Abstract: Suppose I wish to make a graph $G$ triangle-free by removing a small
number of edges.  An obvious obstruction is the presence of a large
set of edge-disjoint triangles, since I must remove one edge from each
triangle. On the other hand, removing all the edges in a maximal set
of edge-disjoint triangles clearly makes $G$ triangle-free.
\emph{Tuza's conjecture} states that the worst-case number of edges
that must be removed is somewhere between these extremes: if $\tau(G)$
is the number of edges that must be removed to make $G$ triangle-free
and $\nu(G)$ is the maximum number of edge-disjoint triangles in $G$,
then Tuza's conjecture states that $\tau(G) \leq 2\nu(G)$. Using the
method of discharging, we show that Tuza's conjecture holds whenever
the maximum average degree $\mathrm{Mad}(G) < 7$. This subsumes
several earlier results and represents the first application of
discharging to this problem.
Sat Mar  2 18:03:12 EST 2013

first_name: Benjamin
last_name: Stucky
email: bwstucky@gmail.com
institution: University of Oklahoma
Talk: Yes
Title: An Algorithmic Approach to Raising the Lower Bound of $R(5,5)$
Abstract: The primary focus of this talk is to explore some basic methods for pushing the lower bound on the classical Ramsey number $R(5,5)$ from $43$ to $44$.  Interestingly enough, we can phrase the problem in the following way: Is it possible to have an arrangement of $43$ people in a room such that no $5$ are \textit{either} mutual acquaintances \textit{or} mutual strangers (a question that perhaps gave rise to the informal description of this problem as the ``Party Problem'').  First, I explore the simple brute force method involving testing $2^{903}$ configurations of $2$-colorings of the edges of $K_{43}$, the complete graph on $43$ vertices.  Next I discuss some simplifications to this approach, including attempting to exploit symmetry in configurations to eliminate graphs to test.  Finally, I discuss the approach involving extending an existing solution on one less vertex, drastically eliminating the number of graphs to be searched to $2^{42}$. Having been unsucces!
 sful in using my algorithm to raise the lower bound of $R(5,5)$, I discuss some rationalizations for my ``failure'' and an argument for $R(5,5)=43$. I will start with a description and history of the problem, continue with a description of my approach, and finish by outlining my results.
Sun Mar  3 22:11:33 EST 2013

first_name: Deepak
last_name: Bal
email: dbal@cmu.edu
institution: Carnegie Mellon University
Talk: Yes
Title: Packing Tree Factors in Random and Pseudo-Random Graphs
Abstract: For a fixed graph $H$ with $t$ vertices, an $H$-factor of a graph $G$ with $n$ vertices is a collection of vertex disjoint (not necessarily induced) copies of $H$ in $G$ covering all vertices of $G$.  We prove that certain pseudo-random and random graphs may have almost all of their edges covered by a collection of edge disjoint $H$-factors in the case where $H$ is a tree.

Joint work with Alan Frieze, Michael Krivelevich and Po-Shen Loh.
Tue Mar  5 12:18:35 EST 2013

first_name: Zhihong
last_name: Chen
email: chen@butler.edu
institution: Butler University
Talk: Yes
Title: Eulerian-connected Graphs
Abstract: A graph $G$ is Eulerian-connected if for any $u$ and $v$ in $V(G)$, $G$ has a spanning $(u,v)$-trail.
A graph $G$ is edge-Eulerian-connected if for any $e'$ and $e''$ in $E(G)$, $G$ has a spanning $(e',e'')$-trail.
For an integer $r\ge 0$, a graph is called $r$-Eulerian-connected if for any $X\subseteq E(G)$ with $|X|\le r$,
and for any $u,\ v \in V(G)$,
$G$ has a spanning $(u,v)$-trail $T$ such that $X\subseteq E(T)$. The $r$-edge-Eulerian-connectivity of a graph can be defined similarly.
 Let $\theta (r)$ be the minimum value of $k$ such
that every $k$-edge-connected graph is $r$-Eulerian-connected.
 Catlin proved that $\theta(0)=4$. In this talk, we shall show that $\theta(r)=4$ for $0\le r\le 2$, and $\theta(r)=r+1$ for $r\ge 3$, and show
 some new results on $r$-edge-Eulerian connectivity.

Wed Mar  6 10:48:53 EST 2013

first_name: David
last_name: Galvin
email: dgalvin1@nd.edu
institution: University of Notre Dame
Talk: Yes
Title: Twin conventions and graph Stirling numbers
Abstract: Recently Griffiths asked (and answered) the question: in how many ways can $n$ sets of twins at a twin convention break into non-empty groups, with no group allowed to contain a pair of twins? One approach to this problem leads naturally to the notion of a ``graph Stirling number of the second kind''. I'll say what we know about these numbers and what we don't, and then use them to address Griffith's question with ``twins'' replaced by ``triplets'', ``quadruplets'', etc. Partly joint work with Do Trong Thanh.
Thu Mar  7 10:51:37 EST 2013

first_name: Steve
last_name: Butler
email: butler@iastate.edu
institution: Iowa State University
Talk: Yes
Title: Throttling zero forcing propagation speed
Abstract: Zero forcing is a game played on a graph that starts with a coloring of the vertices as white and black and at each step any vertex colored black with a unique neighbor colored white ``forces'' the color of the white vertex to be come black.  In this talk we look at what happens when we balance the size of the initial set of vertices colored black and the length of time that it takes for all vertices to be colored black.  We also give an example that shows it is possible in some graphs to slow down the speed of propagation in the graph by choosing larger initial sets.

Joint work with Michael Young.



Sat Mar  9 07:38:44 EST 2013

first_name: Ann
last_name: Darke
email: darkea@bgsu.edu
institution: Bowling Green State University
Talk: no
Title:
Abstract:
Tue Mar 12 01:47:53 EDT 2013

first_name: Zeinab
last_name: Maleki
email: zmaleki@illinois.edu
institution: University of Illinois Urbana-Champaign
Talk: Yes
Title: Some lower bounds for the intersection dimension of graphs
Abstract: The {\it intersection dimension} of a graph with respect to a set of non-negative integers~$L$ is the small number~$l$ for which there is an assignment on the vertices to subsets $A_v \subseteq \{1,\dots, l\}$, such that every two vertices $u,v$ are adjacent if and only if $|A_u \cap A_v|\in L$.
The {\it bipartite intersection dimension} is defined similarly when the conditions are considered only for the vertices in different parts. The {\it absolut dimension} of a graph $G$ is the minimum intersection dimension of $G$ over all sets $L$. Finding graphs with large (bipartite) absolute dimension would have important consequences in the complexity theory.
In this talk, we present some lower bounds for the (bipartite) intersection dimension of a graph with respect to various types $L$ in terms of the minimum rank of graph.
(This is a joint work with Behnaz Omoomi)
Tue Mar 12 21:57:33 EDT 2013

first_name: Sogol
last_name: Jahanbekam
email: sowkam305@gmail.com
institution: University of Illinois at Urbana-Champaign
Talk: Yes
Title: Rainbow Spanning Subgraphs in Edge-Colored Complete Graphs
Abstract: For integers $n$ and $t$, let $r(n,t)$ be the maximum number of colors in an
edge-coloring of the complete graph $K_n$ that does not contain $t$
edge-disjoint rainbow spanning trees.  Let $s(n,t)$ be the maximum number of colors in an edge-coloring of $K_n$ containing no rainbow spanning subgraph with diameter at most $t$.  We determine r(n,t) for
$n>2t+\sqrt{6t-\frac{23}{4}}+\frac52$ and for $n=2t$. We also determine $s(n,t)$ for all integers $n$ and $t$.
This is joint work with Douglas B. West.

Fri Mar 15 13:34:05 EDT 2013

first_name: Ariana
last_name: Angjeli
email: aangjeli@oakland.edu
institution: OAKLAND UNIVERSITY
Talk: Yes
Title: LINEARLY MANY FAULTS IN DUAL-CUBE-LIKE NETWORKS
Abstract: LINEARLY MANY FAULTS IN DUAL-CUBE-LIKE NETWORKS
   ARIANA ANGJELI,EDDIE CHENG AND LASZLO LIPTAK
 ABSTRACT. The dual cube were introduced as better interconnections network than the hypercubes for large scale distributed memory multiprocessors. In this talk we introduce a generalization of these networks, called dual-cube-like networks, which preserve the basic structure of dual cubes and retain many of its topological properties. We investigate structural properties of these networks beyond simple measures such as connectivity. We prove that if up to kn-(k(k+1))/2 vertices are deleted from dual-cube-like n-regular network then the resulting graph will either be connected or will have large components and small components having at most k-1 vertices in total,and this result is sharp for k%E2%89%A4n. As an application we derive additional results such as the cyclic vertex-connectivity and the restricted vertex-connectivity of these networks.


Tue Mar 19 10:48:23 EDT 2013

first_name: Arthur
last_name: Busch
email: art.busch@udayton.edu
institution: University of Dayton
Talk: Yes
Title: Toughness and Hamiltonicity in k-trees
Abstract: A chordal graph is a $k$-tree if every maximal clique has order $k+1$ and every minimal separating set has order $k$.  In this talk we improve a theorem of Broesma, Xiong & Yoshimoto on toughness in $k$-trees and consider some related corollaries.
Tue Mar 19 13:51:57 EDT 2013

first_name: John
last_name: Engbers
email: jengbers@nd.edu
institution: University of Notre Dame
Talk: Yes
Title: Extremal $H$-colorings of forests and trees
Abstract: Given a finite graph $H$, an $H$-coloring of a finite, simple graph $G$ (or graph homomorphism) is a map from the vertices of $G$ to the vertices of $H$ that preserves edge adjacency.  $H$-colorings generalize many important graph theoretic notions, such as proper $q$-colorings (via $H = K_{q}$) and independent sets (via $H$ as an edge with a loop on one endvertex).

Given a family of graphs $\mathcal{G}$ and a graph $H$, which graph(s) in $\mathcal{G}$ have the largest number of $H$-colorings?  We present several results for the family of $n$-vertex forests and the family of $n$-vertex trees.  Numerous open questions remain.

Tue Mar 19 14:01:51 EDT 2013

first_name: Darren
last_name: Parker
email: parkerda@gvsu.edu
institution: Grand Valley State University
Talk: Yes
Title: Multidesigns for Graph-Triples of Order 6
Abstract: We call $T=(G_1,G_2,G_3)$ a \emph{graph-triple of order t} if the $G_i$ are pairwise non-isomorphic graphs on $t$ non-isolated vertices whose edges can be combined to form $K_t$.  If $m \geq t$, we say $T$ \emph{divides} $K_m$ if $E(K_m)$ can be partitioned into copies of the graphs in $T$ with each $G_i$ used at least once, and we call such a partition a \emph{$T$-multidecomposition}.  In this talk, we determine $T$-multidecompositions of complete graphs, where $T$ is a graph-triple of order 6. Moreover, we determine maximum multipackings and minimum multicoverings when $K_m$ does not admit a multidecomposition.
Wed Mar 20 09:22:50 EDT 2013

first_name: David
last_name: Craft
email: craft@muskingum.edu
institution: Muskingum University
Talk: no
Title:
Abstract:

Wed Mar 20 17:14:37 EDT 2013

first_name: ILKYOO
last_name: CHOI
email: ichoi4@illinois.edu
institution: University of Illinois at Urbana-Champaign
Talk: Yes
Title: On Choosability with Separation of Planar Graphs with Forbidden Cycles
Abstract: We study choosability with separation which is a constrained version of list coloring of graphs. A \emph{$(k,d)$-list assignment} $L$ of a graph $G$ is a function that assigns to each vertex $v$ a list $L(v)$ of at least $k$ colors and for any adjacent pair $xy$, the lists $L(x)$ and $L(y)$ share at most $d$ colors. A graph $G$ is $(k,d)$-choosable if there exists an $L$-coloring of $G$ for every $(k,d)$-list assignment $L$. This concept is also known as choosability with separation. We prove that planar graphs without 4-cycles are $(3,1)$-choosable and that planar graphs without 5-cycles and 6-cycles are $(3,1)$-choosable. In addition, we give an alternative and slightly stronger proof that triangle-free planar graphs are $(3,1)$-choosable.


Fri Mar 22 12:24:22 EDT 2013

first_name: David
last_name: Anderson
email: eymiha@gmail.com
institution: Neo Innovation
Talk: no
Title:
Abstract:


Fri Mar 22 14:36:36 EDT 2013

first_name: Brian
last_name: Kell
email: bkell@cmu.edu
institution: Carnegie Mellon University
Talk: no
Title:
Abstract:


Sat Mar 23 13:50:37 EDT 2013

first_name: Nicholas
last_name: Peterson
email: peterson@math.ohio-state.edu
institution: The Ohio State University
Talk: Yes
Title: On Random $k$-Out Graphs with Preferential Attachment
Abstract: We generalize a model of Hansen and Jaworski for random mappings which exhibit preferential attachment to a mapping $[n]\mapsto[n]^k$ - or, equivalently, a random digraph $M_{n,k}^{\alpha}$ with labeled arcs and uniform out-degree $k$. Each vertex starts with some weight $\alpha>0$; each vertex chooses $k$ images, one at a time in a fixed order, with probability of choosing a given vertex proportional to its current weight; and the weight of the chosen vertex increases by 1 before the next decision is made.  The limiting case $M_{n,k}^{\infty}$ is a uniformly random $k$-out digraph with labeled arcs.

We establish a limiting distribution for the vertex connectivity of the graph obtained from $M_{n,k}^{\alpha}$ by ignoring loops, multiple edges, and arc directions. We also seek to answer the question: how fast must $\alpha$ grow (relative to $n$) in order to make the difference between $M_{n,k}^{\alpha}$ and $M_{n,k}^{\infty}$ asymptotically negligible?  Measuring with the total variation distance, we establish $\alpha=\Theta(\sqrt{n})$ as a sharp threshold for this behavior.

This is a joint work with Boris Pittel.
Sat Mar 23 22:50:35 EDT 2013

first_name: Betsie
last_name: Davis
email: davibr03@students.ipfw.edu
institution: IPFW
Talk: no
Title:
Abstract:
Sat Mar 23 22:52:36 EDT 2013

first_name: Theodore
last_name: Eagleson
email: theodoreeagleson@yahoo.com
institution: IPFW
Talk: no
Title:
Abstract:
Sun Mar 24 12:43:44 EDT 2013

first_name: Daniel
last_name: Poole
email: poole@math.osu.edu
institution: The Ohio State University
Talk: Yes
Title: Weak Hamilton Cycles in Random Hypergraphs
Abstract: We say that a hypergraph, $H$, has a weak Hamilton cycle if there is some
cyclic ordering of the vertices of $H$, such that each consecutive pair of
vertices is contained in some hyperedge. We find the sharp threshold for the
existence of a weak Hamilton cycle in the random d-uniform hypergraph,
$H_d(n,m)$, on $n$ vertices with $m$ hyperedges. While the
Erd\H{o}s-R\'{e}nyi graph $G(n,m)=H_2(n,m)$ with high probability develops a
Hamilton cycle when the minimum vertex degree reaches $2$, for $d>2$
$H_d(n,m)$ whp becomes Hamiltonian sooner, when the minimum vertex degree
reaches $1$. Our proofs use hypergraph analogues of Pos\'{a}'s lemma and De
la Vega's Theorem. 
Sun Mar 24 15:03:12 EDT 2013


first_name: Stephanie
last_name: Edwards
email: sedwards@hope.edu
institution: Hope College
Talk: no
Title:
Abstract:
Sun Mar 24 17:20:51 EDT 2013

first_name: Tom
last_name: Mahoney
email: thomas.r.mahoney@gmail.com
institution: University of Illinois at Urbana-Champaign
Talk: Yes
Title: Online sum list coloring of graphs
Abstract: In \textit{online list coloring} (introduced by Zhu and by Schauz in 2009), on each round the set of vertices having a particular color in their lists is revealed, and the coloring algorithm chooses an independent subset to receive that color. A graph $G$ is said to be $f$-paintable for a function $f:V(G)\to\N$ if there is an algorithm to produce a successful coloring when each vertex $v$ is allowed to be presented at most $f(v)$ times.

In 2002 Isaak introduced \textit{sum list coloring} and the resulting parameter called \textit{sum-choosability}. The \textit{online sum-choosability}, or \textit{sum-paintability}, of $G$  is the least $\sum f(v)$ over all functions $f$ such that $G$ is $f$-paintable; this value is denoted by $\chi_{sp}(G)$.

The \textit{generalied theta-graph} $\Theta_{\ell_1,\dots,\ell_k}$ consists of two vertices joined by internally disjoint paths of lengths $\ell_1,\dots,\ell_k$.
Strengthening results of Berliner et al., we show that the sum-paintability of $G$ depends only on the sum-paintability of its blocks, and we prove $\chi_{sp}(K_{2,r})=2r+\min\{ l + m : lm>r \}$.
The \textit{book} $B_r$ is the graph $\Theta_{1,2,\dots,2}$ with $r$ internally disjoint paths of length 2. We prove $\chi_{sp}(B_r)=2r+\min\{ l + m : m(l-m)+{m\choose2} > r \}$.
Using these results, we determine the sum-paintability for all generalized theta-graphs.
Sun Mar 24 17:54:19 EDT 2013

first_name: Axel
last_name: Brandt
email: axel.brandt@ucdenver.edu
institution: University of Colorado Denver
Talk: Yes
Title: 21st Century Cops and Robbers
Abstract: Since it's introduction in 1976, pursuit-evasion games on graphs appear in many forms. The well known cops and robbers game in which a `Cop' attempts to locate a `Robber' on a graph relates to significant theoretical topics such as treewidth.

We consider a variation introduced by Seager in 2012. Each turn, the Cop `probes' a vertex $v$ in the pursuit graph $G$ and receives the Robber's current distance from $v$. The Robber is then permitted to move to any vertex adjacent to his current location, with the exception of $v$. The Cop wins if at any time he is able to determine the Robber's location exactly.

Seager gave a non-constructive proof that on any tree $T$, the Cop can win in at most $|T|-2$ probes. In this talk, we give an explicit strategy for the Cop to win on a tree $T$ in at most $|T|-2$ probes, meeting Seager's bound.

This is joint work with Jennifer Diemunsch, Catherine Erbes, Jordan Legrand, and Casey Moffatt.
Sun Mar 24 18:34:15 EDT 2013

first_name: Jaehoon
last_name: Kim
email: kim805@illinois.edu
institution: University of Illinois at Urbana-Champaign
Talk: Yes
Title: (0,1)-improper coloring of sparse triangle-free graph.
Abstract: A graph $G$ is a $(0,1)$-colorable if $V(G)$ can be partitioned into two sets $V_0$ and $V_1$ so that $G[V_0]$ is an independent set and $G[V_1]$ has maximum degree at most $1$. The problem of verifying whether a graph is $(0,1)$-colorable is NP-complete even in the class of planar graphs of girth 9.

Maximum average degree, $Mad(G)= \max_{H\subset G}\{\frac{2|E(H)|}{|V(H)|}\}$, is a graph parameter measuring how sparse the  graph $G$ is. Borodin and Kostochka showed that every graph $G$ with $Mad(G)\leq \frac{12}{5}$ is $(0,1)$-colorable, thus every planar graph with girth at least $12$ also is $(0,1)$-colorable.

The aim of this talk is to prove that every triangle-free graph $G$ with $Mad(G)\leq \frac{22}{9}$ is $(0,1)$-colorable. We prove the slightly
stronger statement that every triangle-free graph $G$ with $|E(H)|<\frac{11|V(H)|+5}{9}$ for every subgraph $H$ is $(0,1)$-colorable and show that there are infinitely many not  $(0,1)$-colorable graphs $G$ with $|E(G)|=\frac{11|V(G)|+5}{9}$. This is joint work with A. V. Kostochka and Xuding Zhu.
Mon Mar 25 10:41:58 EDT 2013

first_name: Stephen
last_name: Young
email: stephen.young@louisville.edu
institution: University of Louisville
Talk: Yes
Title: An Alon-Boppana result for the normalized Laplacian
Abstract: We prove a Alon-Boppana style bound for the spectral gap of the normalized
Laplacian for general graphs via a bound on the weighted spectral radius of
the universal cover graph.
Mon Mar 25 11:41:22 EDT 2013

first_name: Robert
last_name: Seiver
email: seiverrt@miamioh.edu
institution: Miami University
Talk: Yes
Title: Set Families without Chain-traces
Abstract: Given a family $\cF$ of subsets of a set $X$ and $W\subseteq X$, the {\it trace} of $\cF$ on $W$, denoted by $\cF|_W$ is defined to be  $\cF|_W=\{F\cap W: F\in \cF\}$. If for some $k$-set $W$ in $X$,
$\cF|_W$ contains a maximal chain $C_0\subset C_1\cdots \subset C_k$ with $|C_i|=i$,
then we say that $\cF$ has a {\it $k$-chain-trace}. For fixed $k$ and sufficiently large $n$, Patk\'os  showed that every family $\cF$ of $k$-subsets of $[n]$ with more than $\binom{n-1}{k-1}$ members
has a $k$-chain-trace. The bound is best possible. Patk\'os made a more general conjecture
that for fixed $m\geq k\geq 2$ and sufficiently large $n$, every family $\cF$ of $m$-subsets of $[n]$ with
more than $\binom{n-m+k-1}{k-1}$ members has a $k$-chain-trace. We prove Patk\'os' conjecture and for large $n$ also establish strong stability property of the unique
extremal family.
Mon Mar 25 13:10:27 EDT 2013

first_name: Huseyin
last_name: Acan
email: acan@math.ohio-state.edu
institution: The Ohio State University
Talk: Yes
Title: Evolution of a Random Permutation Graph
Abstract:

Associated with a permutation $\sigma$ of $[n]$, there is a graph $G_{\sigma}$ whose vertex set is $[n]$ and whose edges correspond to the inversions of $\sigma$. Let $\sigma(n,m)$ denote a permutation chosen uniformly at random from all permutations of $[n]$ with $m$ inversions.  We find a growth process of a random permutation in which we obtain $\sigma(n,m)$ after the $m$-th step. We show that the connectedness probability of $G_{\sigma(n,m)}$ is non-decreasing in $m$. We also find the threshold value of the number of edges to be $\frac{6}{\pi^2}n\ln(n)$ for the connectedness of the inversion graph $G_{\sigma(n,m)}$. This is a joint work with Boris Pittel.
Mon Mar 25 14:34:56 EDT 2013

first_name: Jay
last_name: Bagga
email: jbagga@bsu.edu
institution: Ball State University
Talk: no
Title:
Abstract:
Mon Mar 25 15:12:12 EDT 2013

first_name: Daniel
last_name: McDonald
email: dmcdona4@illinois.edu
institution: UIUC
Talk: Yes
Title: On-line rank number of trees
Abstract: A $k$-ranking of a graph $G$ is a labeling of its vertices from $[k]$ such that any nontrivial path whose endpoints have the same label contains a larger label.  The least $k$ for which $G$ has a $k$-ranking is the rank number of $G$, also known as tree depth.  Applications of rankings include VLSI design, parallel computing, and factory scheduling. The on-line ranking problem asks for an algorithm for ranking the vertices of $G$ as they are presented one at a time along with all previously ranked vertices and the edges between (so each vertex is presented as the lone unranked vertex in a partially labeled induced subgraph of $G$ whose final placement in $G$ is not specified).  The on-line rank number of $G$ is the minimum over all such algorithms of the largest label that algorithm can be forced to use.  We give upper and lower bounds on the on-line rank number of trees in terms of maximum degree, diameter, and other structural parameters.
Mon Mar 25 15:38:32 EDT 2013

first_name: Darin
last_name: Stephenson
email: stephenson@hope.edu
institution: Hope College
Talk: no
Title:
Abstract:
Mon Mar 25 17:01:02 EDT 2013

first_name: Peter
last_name: Blanchard
email: peter-blanchard@uiowa.edu
institution: Grinnell College, University of Iowa
Talk: Yes
Title: Coloring in Classical Lattices
Abstract: We prove a Van der Waerden type theorem concerning finite colorings of the Boolean Lattice B_n and the Young Lattice.  This is joint project with Amitava Bhattacharya, Tata Institute of Fundamental Research, Mumbai, India.
Mon Mar 25 18:49:50 EDT 2013

first_name: Lisa
last_name: Espig
email: lespig@andrew.cmu.edu
institution: Carnegie Mellon University
Talk: Yes
Title: Threshold for Zebraic Hamilton Cycles in Random Graphs
Abstract: When studying random graphs, we often want to know the threshold for the emergence of a particular structure. In other words, in the Erd\"os-R\'enyi random graph model $G_{n,p}$ - where we consider a graph on $n$ vertices with each edge appearing randomly with probability $p$, how large must $p$ be in order to guarantee a matching in the graph with high probability? This and other thresholds are well-studied. Here we find the threshold for a different kind of structure. Namely, we will find the threshold for which a randomly 2-colored random graph contains a zebraic Hamilton cycle - one whose edges alternate between the two colors.
Mon Mar 25 20:09:58 EDT 2013

first_name: Bernard
last_name: Lidick%C3%BD
email: lidicky@illinois.edu
institution: University of Illinois at Urbana-Champaign
Talk: Yes
Title: Coloring planar graphs with 4-triangles
Abstract: A sharpening of Gr\" otzsch Theorem, the Gr\" unbaum-Axenov
Theorem, states that
every planar graph with at most three triangles is 3-colorable.
It is best possible since not all planar graphs with four triangles
are 3-colorable.
In this talk, we discuss 3-colorability of planar graphs
with four triangles.

This is joint work with O. V. Borodin, Z Dvo\v{r}\'ak, A. Kostochka, and M. Yancey
Mon Mar 25 20:30:37 EDT 2013

first_name: Luke
last_name: Nelsen
email: nelsenll@miamioh.edu
institution: Miami University (OH) -- Graduate Student
Talk: no
Title:
Abstract:
Mon Mar 25 22:27:32 EDT 2013

first_name: Bayli
last_name: Palmer
email: Palmerbb@miamioh.edu
institution: Miami University
Talk: no
Title:
Abstract:
Mon Mar 25 23:31:57 EDT 2013

first_name: Uta
last_name: Ziegler
email: uta.ziegler@wku.edu
institution: Western Kentucky University
Talk: Yes
Title: Embedding of 4-regular planar graphs on a grid
Abstract: The author and her collaborators have developed methods to embed 4-regular planar graphs into a 2-dimensional grid while preserving the topology (i.e the cyclic order of vertices at each vertex). We call a 4-regular planar graph algebraic if it can be reduced to a trivial graph (consisting of one 4-regular vertex) by the repeated collapses of digons. Here we present a special case of our algorithm for algebraic 4-regular graphs, where it can be shown that the length of a constructed embedding is linear in the number of vertices of the original graph. These graph embedding results can be applied to obtain results about the rope length of knots.
Tue Mar 26 08:09:57 EDT 2013

first_name: Nathan
last_name: Graber
email: grabernt@miamiOH
institution: Grad Student at Miami
Talk: no
Title:
Abstract:



\end{document}