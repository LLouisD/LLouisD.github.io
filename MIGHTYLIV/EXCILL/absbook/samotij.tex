\speaker{Wojciech Samotij}{wojteksa@gmail.com}{University of Cambridge and Tel Aviv University}
\title{Typical structure of graph homomorphisms}
\endtitle

Given two graphs $G$ and $H$, a \emph{graph homomorphism} from $G$ to $H$ is a mapping of $V(G)$ to $V(H)$ that maps edges of $G$ to edges of $H$. Counting and describing the typical structure of homomorphisms between graphs have many interesting aspects and a surprising number of applications. Numerous models in statistical mechanics and questions in extremal graph theory can be phrased in these terms.

In this talk, we will focus our attention on the case when $H$ is some small fixed graph, which can be thought of as a generalization of graph coloring. For many natural classes $\mathcal{C}$ of graphs, one observes the following phenomenon: A typical homomorphism from every $G \in \mathcal{C}$ to $H$ is very rigid, exhibiting strong spatial correlations. We will discuss several examples of this phenomenon, focusing on two particular settings: (i) $G$ is (a large subbox of) the $d$-dimensional integer grid and (ii) $G$ is a regular bipartite graph with strong expansion properties.

Joint work with Ron Peled (Tel Aviv University) and Amir Yehudayoff (Technion).
