\speaker{Penny	Haxell}{pehaxell@math.uwaterloo.ca}{University of Waterloo}
\title{Extremal hypergraphs for packing and covering}
\endtitle

A {\it packing} (or {\it matching}) in a hypergraph $H$ is a set of pairwise 
disjoint edges of $H$. A {\it cover} of $H$ is a set $C$ of vertices that
meets all edges of $H$. A famous open problem known as Ryser's
Conjecture states that 
any $r$-partite $r$-uniform hypergraph should have a cover of size at
most $(r-1)\nu(H)$, where $\nu(H)$ denotes the size of a largest
packing in $H$. This was proved by Aharoni in 2001 for the case
$r=3$. Here we show that if equality holds in this case then $H$
belongs to a special class of hypergraphs we call ``home base
hypergraphs''. To prove this we need to establish some
auxiliary results on connectedness of the matching complex of bipartite graphs.

Joint work with L. Narins and T. Szab\'o. 