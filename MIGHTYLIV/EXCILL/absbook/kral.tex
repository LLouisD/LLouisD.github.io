\speaker{Daniel	Král'}{d.kral@warwick.ac.uk}{University of Warwick}
\title{Quasirandomness of permutations}
\endtitle

A systematic study of large combinatorial objects has recently leds
to discovering many connections between discrete mathematics and
analysis. In this talk, we apply analytic methods to permutations.
In particular, we associate every sequence of permutations
with a measure on a unit square and show the following:
if the density of every 4-element subpermutations in a permutation $p$
is $1/4!+o(1)$, then the density of every k-element subpermutation
is $1/k!+o(1)$. This answers a question of Graham whether quasirandomness
of a permutation is captured by densities of its 4-element subpermutations.

The result is based on a joint work with Oleg Pikhurko.

