\speaker{Tao	Jiang}{jiangt@miamioh.edu}{Miami University}
\title{Hypergraph Tur\'an and Ramsey results on linear cycles}
\endtitle

A hypergraph is linear if every two edges intersect in at most
one vertex. The $k$-expansion of a $q$-graph $H$ is obtained by
expanding edges of $H$ into $k$-sets using disjoint sets of new vertices.
Let $C^{k}_\ell$ be the $k$-expansion of a $2$-uniform 
cycle of length $\ell$.
For all $k\geq 5$, $\ell\geq 3$ and  large $n$, 
we determine the precise value of the Tur\'an number
$ex_k(n,C^k_\ell)$. For odd $\ell=2t+1$, the answer matches
the Tur\'an number of a $t+1$-matching. For $k\geq 5$ and large $n$ the result
generalizes the Erd\H{o}s-Ko-Rado theorem and Erd\H{o}s'
result on hypergraph matchings and confirms a conjecture of
Mubayi and Verstra\"ete. In general, we show that
if $q\geq 3$ and $k\geq 2q-1$ and $H$
is the $k$-expansion of a subgraph of a $q$-uniform hypertree, 
then $ex_k(n,H)=(\sigma(H)-1+o(1))\binom{n}{k-1}$,
where $\sigma(H)$ is the minimum size of a $1$-cross-cut of $H$.
Our main tool is the delta system method. This is joint work with Z. F\"uredi.

\medskip

We also consider the linear Tur\'an number $ex_L(n,C^k_\ell)$
which is the largest size of a linear $k$-graph on $[n]$
not containing $C^k_\ell$. Extending Bondy-Simonovits' result on even cycles, 
we show that $ex_L(n,C^k_{2m})=O(n^{1+\frac{1}{m}})$. We then use this to show that $R(C^k_{2m}, K^k_n)=O(n^{\frac{m}{m-1}})$.
This is joint work with C. Collier-Cartaino.

