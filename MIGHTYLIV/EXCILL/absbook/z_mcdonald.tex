\speakerposter{Daniel McDonald}{dmcdona4@illinois.edu}{UIUC}
\titleposter{On-line rank number of trees}
\endtitle

A $k$-ranking of a graph $G$ is a labeling of its vertices from $\{1,\ldots,k\}$ such that any path whose endpoints have the same label contains a larger label.  The least $k$ for which $G$ has a $k$-ranking is the rank number of $G$, also known as tree depth.  Applications of rankings include VLSI design, parallel computing, and factory scheduling. The on-line ranking problem asks for an algorithm for ranking the vertices of $G$ as they are presented one at a time along with all previously ranked vertices and the edges between (so each vertex is presented as the lone unranked vertex in a partially labeled induced subgraph of $G$ whose final placement in $G$ is not specified).  The on-line rank number of $G$ is the minimum over all such algorithms of the largest label that algorithm can be forced to use.  We give upper and lower bounds on the on-line rank number of trees in terms of maximum degree, diameter, and other structural parameters.