\speaker{Paul N. Balister}{pbalistr@memphis.edu}{University of Memphis}
\title{Controllability, matchings, and the Karp-Sipser algorithm on random graphs}
\endtitle
In a 2011 article in Nature magazine, Liu, Slotine, and Barabási derived an asymptotic estimate for the number of nodes needed to ``control'' a generic random dynamical network. This is effectively equivalent to finding the size of a maximal matching in a random bipartite graph. The random graphs considered are chosen uniformly from the set of all bipartite graphs with a given fixed degree distribution, and the results therefore depend on these distributions. The results presented in Nature make use of the ``cavity method'', which is inherently non-rigorous. Indeed, these results are false except for very special choices of degree distribution. We derive the correct result for a wide variety of degree distributions; the (rigorous) proof proceeding by an analysis of the Karp-Sipser algorithm. The results also allow for a generalization of work by Bohman and Frieze in which the non-bipartite case was considered. 
Joint work with Stefanie Gerke.
