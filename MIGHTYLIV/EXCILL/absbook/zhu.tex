\speaker{Xuding	Zhu}{xudingzhu@gmail.com}{Zhejiang Normal University}
\title{Total weight choosability of graphs}
\endtitle
A (proper) total weighting of a graph $G$ is a mapping
$\phi: V(G) \cup E(G) \to R$ such that any edge $uv$ of $G$,
$\sum_{e \in E(u)}\phi(e) + \phi(u) \ne \sum_{e\in E(v)}\phi(e) + \phi(v).$
A total weighting $\phi$ with $\phi(v)=0$ for all vertices $v$ is
called an   edge weighting.
The well-known 1-2-3 conjecture says that every
graph with no isolated edges has a   $3$-edge weighting,
i.e., an edge weighting using weights
1,2,3.  Berry, Dalal, McDiarmid, Reed and
Thomason proved that  every
graph with no isolated edges has a  $30$-edge weighting.
This upper bound $30$  was  reduced subsequently to $16$
(by Addario-Berry, Dalal and Reed), $
13$ (by Wang and Yu) and $5$ (by Kalkowski, Karosnki and Pfender).
Przyby{\l}o and  Wo\'{z}niak  conjectured  that every graph has a  $2$-total weighting, and
proved that every graph has a   $11$-total weighting.
Kalkowski made a breakthrough and proved that   every graph $G$
has a  total weighting $\phi$ with $\phi(v) \in \{1,2\}$ for
$v \in V(G)$ and $\phi(e) \in \{1,2,3\}$ for $e \in E(G)$.
This talk discusses the list version of total weighting. A graph $G$ is
called $(k,k')$-choosable if for any list assignment $L$ which assigns to
each vertex $k$ permissible weights, and to each edge $k'$ permissible weights, there
is a  total weighting $\phi$ with $\phi(v) \in L(v)$ and $\phi(e) \in L(e)$,
for $v \in V(G)$ and $e \in E(G)$. Wong and Zhu conjectured that every graph with no isolated edges
is $(1,3)$-choosable, and every graph is $(2,2)$-choosable. This talk survey some progress on these conjectures.
In particular, we shall prove that every graph is $(2,3)$-choosable.