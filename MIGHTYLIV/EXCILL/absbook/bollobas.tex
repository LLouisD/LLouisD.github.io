\speaker{Béla	Bollobás}{b.bollobas@dpmms.cam.ac.uk}{The University of Memphis and Cambridge University}
\title{Bootstrap Percolation -- Problems, Results and Conjectures}
\endtitle

Bootstrap percolation is one of the simplest cellular automata around:
it can be viewed as a very simple model of the spread of an infection,
and is also closely related to the Ising model. In the past three decades,
much work has been done on bootstrap percolation on finite grids of a
fixed dimension, in which the initially infected set $A$ is obtained
by selecting its vertices at random, with the same probability $p$,
independently of all other choices. The focus has been on the {\em critical
probability} $p_c$, the value of $p$ at which the probability of {\em percolation}
(eventual full infection)  is $1/2$, say.

\vspace{10pt}
\noindent
Although bootstrap percolation used to be the hunting ground of statistical
physicists and probabilists only, it really belongs to probabilistic combinatorics.
My aim in this talk is to review of some of the basic results concerning
critical probabilities due to Aizenman, Lebowitz, Schonman, Cerf, Cirillo,
Manzo, Holroyd and others, and say a few words about the more recent results
proved by Balogh, Morris, Duminil-Copin and myself. I also hope to mention
some very recent results I have obtained with Holmgren, Smith, Uzzell and Balister
on the time a random initial set takes to percolate. In addition to the `proper'
results, I shall mention several lightweight problems and difficult conjectures.