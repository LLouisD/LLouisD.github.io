\speakerposter{John Engbers}{jengbers@nd.edu}{University of Notre Dame}
\titleposter{Extremal $H$-colorings of graphs}
\endtitle

An \textit{$H$-coloring} of a finite, simple graph $G$ is a map from the vertices of $G$ to the vertices of a finite graph $H$ (without multiple edges, but possibly with loops) that preserves edge adjacency.  $H$-colorings generalize many important graph theoretic notions, such as proper $q$-colorings (via $H = K_{q}$) and independent sets (via $H$ as an edge with a loop on one endvertex).

Consider the following extremal graph theory question: for a given $H$, which graph on $n$ vertices with minimum degree $\delta$ has the largest number of $H$-colorings?  In this poster I prove that the answer for many $H$ is the complete bipartite graph $K_{\delta, n-\delta}$.  I also provide graphs $H$ where disjoint copies of $K_{\delta+1}$ or disjoint copies of $K_{\delta, \delta}$ yield the largest number of $H$-colorings, and show that for $\delta=1$ and $\delta=2$ (and $n$ sufficiently large) these are the only possible extremal graphs.