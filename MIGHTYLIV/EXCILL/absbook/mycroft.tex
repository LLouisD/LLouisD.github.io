\speaker{Richard Mycroft}{r.mycroft@bham.ac.uk}{Birmingham University}
\title{Perfect Matchings in Hypergraphs}
\endtitle
Any graph $G$ on $n$ vertices with $\delta(G) \geq n/2$ admits a perfect
matching, and the proof of this fact is elementary. However, it wasn't
until 2006 that R\"odl, Ruci\'nski and Szemer\'edi proved an analogue of
this theorem for uniform hypergraphs, showing that any sufficiently
large $k$-graph $H$ on $n$ vertices with minimum codegree $\delta(H)
\geq n/2 - C$ admits a perfect matching (they gave the exact value of
$C$, which depends on $n$). This condition is easily shown to be best
possible by a simple lattice-based construction, and since then the
focus of attention has largely moved to vertex degree conditions (for
instance the vertex degree which forces a perfect matching remains an
important open problem).

Returning to codegree conditions, our main theorem shows that any
$k$-graph $H$ on $n$ vertices with $\delta(H) \geq n/k-o(n)$ must either
contain a perfect matching or be close to a graph from one of two
families of constructions, `space barriers' and `divisibility barriers'.
Members of the first family have minimum codegree at most $n/k$, so the
stronger assumption that $\delta(H) \geq n/k +o(n)$ eliminates this
possibility. The latter family of constructions includes the extremal
example for the theorem of R\"odl, Ruci\'nski and Szemer\'edi mentioned
above; its members are characterised by the sizes of intersections of
edges with the parts of some partition $\mathcal{P}$ of $V(H)$ being
members of some fixed sublattice of $\mathbb{Z}^d$, where $d$ is the
number of parts of $\mathcal{P}$. In fact this theorem holds even under
the weaker assumption that `many' $(k-1)$-tuples have at least this
codegree, rather than all.

In this talk, I will explain this result in more detail, and describe a
number of consequential results which follow from it. These include the
following:
\begin{itemize}
\item A multipartite version of the Hajnal-Szemer\'edi theorem, proving
a conjecture of Fischer (except for a known family of counter-examples).
\item The minimum codegree threshold which guarantees a perfect
tetrahedron packing in a 4-graph, answering a question of Pikhurko.
\item An exact characterisation of 3-graphs with $\delta(H) \geq n/3 +
o(n)$ with no perfect matching: such $k$-graphs must admit a partition
of $V(H)$ into parts $A$ and $B$ so that every edge intersects $A$ in an
even number of vertices. Broadly similar results for $k$-graphs with $k
\geq 4$ can also be obtained, although the naive generalisation of the
3-graph case fails.
\item New results on the complexity of the perfect matching decision
problem for $k$-graphs of large minimum codegree, a problem considered
by Karpi\'nski, Ruci\'nski and Szymanska.
\end{itemize}
I will also outline potential avenues for further development of this
approach.